\documentclass{article}
\usepackage{amsmath}
\usepackage{cite}
\usepackage{caption}
\usepackage[a4paper, total={6in, 8in}]{geometry}
\usepackage{algorithm}
\usepackage{algorithmic}
\usepackage{tikz}
\usetikzlibrary{shapes,arrows,chains}
\usepackage{fancyhdr}
\usepackage{subcaption}
\usepackage{hyperref}

\begin{document}
\thispagestyle{fancy}
\fancyhf{}
\rhead{06-03-2017}
\lhead{Multiply 2-digit numbers}
\chead{Aditya Jhawar}

\tikzstyle{calc}=[rectangle,draw,fill=blue!20,text width=4.5cm,text centered,rounded corners,minimum height=3em,minimum width=10em]
\tikzstyle{line}=[draw,-latex']
\tikzstyle{cloud}=[ellipse,draw,fill=red!20,minimum height=2em]

\section{Introduction}
\href{https://en.wikipedia.org/wiki/Vedic_period}{Vedic period} was one of the best times of Indian civilisation and many new advancements were made in all spheres of life.Mathematics,was particularly, a subject of interest. The \href{https://en.wikipedia.org/wiki/Atharvaveda}{Atharva Veda},~\cite{Veda} is one of the four Vedas and contains knowledge abouut everday procedures.It is one of the oldest scriptures pertaining to our civilisation.The below method of multiplication has been derived from the Atharva Veda.~\cite{commentary}

\section{Pseudocode}
\label{pseudocode}
\begin{algorithm}
\caption{\large\textbf{Multiplication of two numbers}}
\begin{algorithmic}
\REQUIRE Two 2-digit numbers
\STATE $x \leftarrow first\_number$
\STATE $y \leftarrow second\_number$
\STATE $n1 \leftarrow x/10$
\STATE $m1 \leftarrow x\%10$
\STATE $n2 \leftarrow y/10$
\STATE $m2 \leftarrow y\%10$
\IF {$n1 == n2$ \AND $m1+m2 == 10$}
	\STATE $mul \leftarrow CASE1(n1,m1,m2)$
\ENDIF
\IF {$n1+n2 == 10$ \AND $m1 == m2$}
	\STATE $mul \leftarrow CASE2(n1,n2,m1)$
\ENDIF
\end{algorithmic}
\end{algorithm}

\begin{algorithm}
\caption{\large\textbf{CASE1}}
\begin{algorithmic}
\label{case1}
\REQUIRE 3 numbers from the calling function
\STATE $a \leftarrow n1*(n1+1)$
\STATE $b \leftarrow m1*m2$
\STATE return $a*100+b$
\end{algorithmic}
\end{algorithm}

\begin{algorithm}
\caption{\large\textbf{CASE2}}
\begin{algorithmic}
\label{case2}
\REQUIRE 3 numbers from the calling function
\STATE $a \leftarrow n1*n2+m1$
\STATE $b \leftarrow m1*m1$
\STATE return $a*100+b$
\end{algorithmic}
\end{algorithm}

\section{Examples}
Let x denote the first two digits and y denote the last two digits of the resulting 4- digit number from the multiplication.

\begin{figure}[H]
\caption{Examples}
\begin{subfigure}{\textwidth}
\centering
\caption{Flowchart illustrating 66x64}
\begin{tikzpicture}[node distance=2cm,auto]
	\node [cloud] (a) {Start};
	\node [calc,below of=a] (c) {x=6*(6+1)=42};
	\node [calc,below of=c] (d) {y=6*4=24};
	\node [calc,below of=d] (e) {Result=4224};
	\node [cloud,below of=e] (f) {End};
	\path [line] (a)--(c);
	\path [line] (c)--(d);
	\path [line] (d)--(e);
	\path [line] (e)--(f);
\end{tikzpicture}
\end{subfigure}

\begin{subfigure}{\textwidth}
\centering
\caption{Flowchart illustrating 74x34}
\begin{tikzpicture}[node distance=2cm,auto]
	\node [cloud] (a) {Start};
	\node [calc,below of=a] (c) {x=3*7+4=25};
	\node [calc,below of=c] (d) {y=4*4=16};
	\node [calc,below of=d] (e) {Result=2516};
	\node [cloud,below of=e] (f) {End};
	\path [line] (a)--(c);
	\path [line] (c)--(d);
	\path [line] (d)--(e);
	\path [line] (e)--(f);
\end{tikzpicture} 
\end{subfigure}
\end{figure}

\section{Explanations}
\subsection{How to do it mentally}
The process of doing the calculation mentally has been given below.It borrows from section \ref{pseudocode}.
\begin{enumerate}
\item Look at the two numbers
\item If their one's digits add upto 10 and their ten's digits are same,we got to step 3 else go to step 5.
\item Multiply the ten's digit with its succeeding number.This gives the ffirst two digits of the product.
\item Multiply their one's digits. This gives the last two digits and we have the complete number.
\item If their ten's digits add upto 10 and their one's digits are same,we got to step 6.\footnote{If neither of the cases is true then our method fails}
\item Multiply their ten's digit and add the result to the one's digit.This gives the first two digits of the product.
\item Multiply their one's digits to get the last two digits of the product.
\item Hence we have the complete number.\footnote{CAUTION:If on multiplying we obtain a single digit then we should consider the second digit to be 0.}
\end{enumerate}

\subsection{Advantages}
The above method is advantageous due to the following reasons:
\begin{itemize}
\item Multiplication is performed on one digit numbers,which is elementary.
\item No tedious additions
\end{itemize}

\subsection{Algebraic}
x denotes the first number.\\
y denotes the second number.\\
n1 denotes the ten's digit of the first number.\\
n2 denotes the ten's digit of the second number.\\
m1 denotes the one's digit of the first number.\\
m2 denotes the one's digit of the second number.\\
\begin{table}[h!]
\begin{tabular}{l|l|l}
\hline
\textbf{Case} & \textbf{Description} & \textbf{Formula}\\
\hline
1 & Ten's digit is same and one's digit add to 10 & n1*(n1+1)*100+m1*m2\\
\hline
2 & Ten's digits add to 10 and one's digit is same & (n1*n2+m1)*100+m1*m1\\
\end{tabular}
\end{table}

\subsubsection{Case 1}
As shown in algorithm\ref{case1}
\begin{align*}
x*y&=(n1*10+m1)*(n2*10+m2)\\
&=n1*n2*100+m1*n2*10+m2*n1*10+m1*m2\\
&=n1*n1*100+(m1+m2)*n1*10+m1*m2\tag*{n1=n2}\\
&=n1*n1*100+n1*100+m1*m2\tag*{m1+m2=10}\\
&=n1*(n1+1)*100+m1*m2\\
\end{align*}


\subsubsection{Case 2}
As shown in algorithm\ref{case2}
\begin{align*}
x*y&=(n1*10+m1)*(n2*10+m2)\\
&=n1*n2*100+n1*m2*10+n2*m1*10+m1*m2\\
&=n1*n2*100+(n1+n2)*m1*10+m1*m1\tag*{m1=m2}\\
&=n1*n2*100+m1*100+m1*m1\tag*{n1+n2=10}\\
&=(n1*n2+m1)*100+m1*m1\\
\end{align*}


\section{Generalisation}
We can generalise it for the product of two n-digit numbers.
\begin{itemize}
\item Take two n-digit numbers.
\item If first  digits are same and the last (n-1) digits add to 10\textsuperscript{n} then
	\begin{itemize}
	\item First 2 digits are given by the product of the n\textsuperscript{th} digits
	\item remaining digits are given by the product of the resulting n-1 digit numbers
	\end{itemize}
\item If first n-1 digits add to 10\textsuperscript{n} and the last digits are same then
	\begin{itemize}
	\item Last two digits are given by the product of the last digits
	\item First digits are given by the product of the resulting n-1 digit numbers and adding the last digit to the product obtained
	\end{itemize}
\item Hence, the product is obtained
\end{itemize}
\bibliography{mybib}
\bibliographystyle{plain}
\end{document}\grid
